\documentclass[twoside,a4paper]{article}
\usepackage{geometry}
\usepackage{ctex, hyperref}
\geometry{margin=1.5cm, vmargin={0pt,1cm}}
\setlength{\topmargin}{-1cm}
\setlength{\paperheight}{29.7cm}
\setlength{\textheight}{25.3cm}

% useful packages.
\usepackage{amsfonts}
\usepackage{amsmath}
\usepackage{amssymb}
\usepackage{amsthm}
\usepackage{enumerate}
\usepackage{graphicx}
\usepackage{multicol}
\usepackage{fancyhdr}
\usepackage{layout}

% some common command
\newcommand{\dif}{\mathrm{d}}
\newcommand{\avg}[1]{\left\langle #1 \right\rangle}
\newcommand{\difFrac}[2]{\frac{\dif #1}{\dif #2}}
\newcommand{\pdfFrac}[2]{\frac{\partial #1}{\partial #2}}
\newcommand{\OFL}{\mathrm{OFL}}
\newcommand{\UFL}{\mathrm{UFL}}
\newcommand{\fl}{\mathrm{fl}}
\newcommand{\op}{\odot}
\newcommand{\Eabs}{E_{\mathrm{abs}}}
\newcommand{\Erel}{E_{\mathrm{rel}}}

\begin{document}

\pagestyle{fancy}
\fancyhead{}
\lhead{褚朱钇恒 (3200104144)}
\chead{Numerical PDE homework \#3}
\rhead{2023.5.5}

\section{Exercise 11.9}

验证real positivity,由$||T||\ge \inf\{M\ge 0\}=0\Rightarrow ||T||\ge 0$

验证point separation,$||T||=0\Rightarrow \forall x,||Tx||\le M||x||=0\Rightarrow T=0$

验证absolute homogeneity,$\forall a\in \mathbb{F},v\in \mathcal{V}(\mathbb{F}^n,\mathbb{F}^m)$

有$||av||=\inf\{M\ge 0:\forall x\in \mathbb{F}^n,|avx|=|a||vx|\le M|x|\}$

若$|a|=0,||av||=0=|a|||v||$,否则,有$||av||=\inf{M\ge 0:\forall x\in \mathbb{F}^n,|vx|\le \frac{M}{|a|}|x|}$,也有$||av||=|a|||v||$

\

验证triangle inequality,$\forall u,v\in \mathcal{V}(\mathbb{F}^n,\mathbb{F}^m)$


$||u+v||=\inf\{M\ge 0:\forall x\in \mathbb{F}^n,|(u+v)x|=|ux+vx|\le M|x|\}$

设$||u||=M_1,||v||=M_2$,则有$\forall x\in \mathbb{F}^n,|ux|\le M_1|x|,|vx|\le M_2|x|\Rightarrow |ux|+|vx|\le (M_1+M_2)|x|$

有$||u+v||\le M_1+M_2=||u||+||v||$

所以有$||\cdot||$是一个范数

\section{Exercise 11.13}
验证non-negativity,由范数的非负性,$d(S,T)=||S-T||\ge 0$

验证identity of indiscernibles,$d(S,T)=0\Leftrightarrow ||S-T||=0 \Leftrightarrow $

$\forall x |(S-T)x|=0=|Sx-Tx|\Leftrightarrow \forall x Sx=Tx \Leftrightarrow S=T$

\

验证symmetry,$d(S,T)=||S-T||=||T-S||=d(T,S)$

验证triangle inequality,$d(A,B)=||A-B||\le ||A-C||+||C-B||=d(A,C)+d(C,B)$

\section{Exercise 11.16}
real positivity,point separation,absolute homogeneity显然,下证triangle inequality:

\begin{align}
  ||u+v||^2&= \sum_{j=1}^n||ue_j+ve_j||^2\\
  &= \sum_{j=1}^n(||ue_j||^2+||ve_j||^2+2\langle ue_j,ve_j\rangle)\\
  &=||u||^2+||v||^2+2\sum_{j=1}^n\langle ue_j,ve_j\rangle(\mbox{由内积的性质})\\
  &\le ||u||^2+||v||^2+2\sqrt{\sum_{j=1}^n||ue_j||^2||ve_j||^2}(\mbox{由柯西不等式})\\
  &\le ||u||^2+||v||^2+2\sqrt{\sum_{j=1}^n||ue_j||^2\sum_{j=1}^n||ve_j||^2}\\
  &\le ||u||^2+||v||^2+2\sqrt{||u||^2||v||^2}\\
  &\le ||u||^2+||v||^2+2||u||||v||\\
  &\le (||u||+||v||)^2\\
\end{align}

两边同时开平方即可得$||u+v||\le ||u||+||v||$


\section{Exercise 11.20}

由11.18有$|TSx|\le|T||Sx|$

则$|TS|^2=\sum_{j=1}^k||TSe_j||^2\le\sum_{j=1}^k(|T||Se_j||)^2=|T|^2\sum_{j=1}^k(|Se_j||)^2=|T|^2|S|^2$

两边同时开平方即可得$|TS|\le|T||S|$

\section{Exercise 11.36}

设$X=PJP^{-1}$,$J$为约当标准型,$P$为可逆矩阵,则有$det(X)=det(PJP^{-1})=det(P)det(J)det(P^{-1})=det(J)$
且$trace(X)=trace(PJP^{-1})=trace(P)trace(J)trace(P^{-1})=trace(J)$

设X的特征值为$\lambda_1,\dots,\lambda_n$,则有$trace(X)=\Sigma_{i=1}^N\lambda_i$

所以有$det(e^{X})=e^{J}=e^{\Sigma_{i=1}^N\lambda_i}=e^{trace(X)}$
\end{document}