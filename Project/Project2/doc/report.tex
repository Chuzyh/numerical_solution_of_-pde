\documentclass[12]{article}%12pt即为*四号字
\usepackage{ctex}%引入中文包
\usepackage{graphicx}%插入图片的包
\usepackage{geometry}%设置A4纸页边距的包
\usepackage{url}
\usepackage{stfloats}
\usepackage{float}
\usepackage{amssymb}
\usepackage{listings}
\geometry{left=3.18cm,right=3.18cm,top=2.54cm,bottom=2.54cm}%设置页边距
\linespread{1}%设置行间距


\begin{document}
\begin{center}
    \LARGE\songti\textbf{微分方程数值解Project2作业报告} \\%标题
    \large\kaishu\textbf{褚朱钇恒\qquad 3200104144}%一般是我的姓名
\end{center}
\section{运行说明}
    本项目需要调用\verb|jsoncpp|与\verb|eigen3|库,故请在运行此项目前安装好这两个包。
    
    您可以使用以下命令进行安装:

    \begin{lstlisting}
        sudo apt-get install libeigen3-dev
        sudo apt-get install libjsoncpp-dev
    \end{lstlisting}

    在\verb|project|目录下使用\verb|make|命令即可编译运行整个项目并得到实验报告。

\section{程序设计思路}
实现多重网格法计算的相关代码在头文件\verb|multigrid.h|中,其中主要设计了template <int dim> class Multigrid\_Method

根据dim的取值(1或2),Multigrid\_Method类将被特化成求解一维或二维的边值问题的类。

\subsection{函数}
\begin{itemize}
    \item 构造函数:需要边界条件,限制算子,插值算子,迭代算法,停止条件,网格粗细,初始解,停止条件参数,最粗网格作为参数
    \item \verb|laplace|:对一个网格上的数值作拉普拉斯变换
    \item \verb|restriction|:限制算子,实现了injection和full\_weighting两种算法
    \item \verb|interpolation|:插值算子,实现了linear和quadratic两种算法
    \item \verb|jacobi|:带权的雅各比迭代,其中参数$\omega=\frac{2}{3}$
    \item \verb|Vcycle|与\verb|FMG|:两种迭代算法
    \item \verb|accuracy|:计算当前解的相对精度
    \item \verb|error\_norm|:计算当前解的误差范数
    \item \verb|residual\_norm|:计算当前解的残差范数
    \item \verb|solve|:制定$v_1$与$v_2$并进行求解
\end{itemize}

\subsection{参数与变量说明}
\subsubsection{bound\_conditon}
实现了三种边界条件:Dirichlet, Neumann,mixed。

其中,纯Neumann边界条件由于没有唯一解,计算时给定了区域中心点的具体点值使解唯一。

mixed条件在一维时为左端点满足Dirichlet条件,右端点满足Neumann边界条件,二维时为上下边界满足Dirichlet条件,左右边界满足Neumann边界条件

\subsubsection{stopping\_criteria}
实现了两种条件:max\_iteration,rela\_accuracy.

前者为限制最大迭代次数为st\_parm,后者为迭代至相对误差小于st\_parm。

但为了避免死循环和减少不必要的迭代,当迭代时误差的收敛速度小于1.01时,也会停止迭代。
\section{程序测试结果}
我选择的测试用的函数为:
\begin{itemize}
    \item $f(x,y)=e^{x+sin(y)}$
    \item $f(x,y)=sin(3x+3y)$
    \item $f(x,y)=e^(x^3+y^3)$
\end{itemize}

\end{document} 