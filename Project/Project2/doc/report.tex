\documentclass[12]{article}%12pt即为*四号字
\usepackage{ctex}%引入中文包
\usepackage{graphicx}%插入图片的包
\usepackage{geometry}%设置A4纸页边距的包
\usepackage{url}
\usepackage{stfloats}
\usepackage{float}
\usepackage{amssymb}
\usepackage{listings}
\geometry{left=3.18cm,right=3.18cm,top=2.54cm,bottom=2.54cm}%设置页边距
\linespread{1}%设置行间距


\begin{document}
\begin{center}
    \LARGE\songti\textbf{微分方程数值解Project2作业报告} \\%标题
    \large\kaishu\textbf{褚朱钇恒\qquad 3200104144}%一般是我的姓名
\end{center}
\section{运行说明}
    本项目需要调用\verb|jsoncpp|与\verb|eigen3|库,故请在运行此项目前安装好这两个包。
    
    您可以使用以下命令进行安装:

    \begin{lstlisting}
        sudo apt-get install libeigen3-dev
        sudo apt-get install libjsoncpp-dev
    \end{lstlisting}

    在\verb|project|目录下使用\verb|make|命令即可编译运行整个项目并得到实验报告。(测试数据较多,可能需要大约10min的时间)

    如果您只需要渲染文档,可以使用\verb|make report|命令

\section{程序设计思路}
实现多重网格法计算的相关代码在头文件\verb|multigrid.h|中,其中主要设计了template <int dim> class Multigrid\_Method

根据dim的取值(1或2),Multigrid\_Method类将被特化成求解一维或二维的边值问题的类。

\subsection{函数}
\begin{itemize}
    \item 构造函数:需要边界条件,限制算子,插值算子,迭代算法,停止条件,网格粗细,初始解,停止条件参数,最粗网格作为参数
    \item \verb|laplace|:对一个网格上的数值作拉普拉斯变换
    \item \verb|restriction|:限制算子,实现了injection和full\_weighting两种算法
    \item \verb|interpolation|:插值算子,实现了linear和quadratic两种算法
    \item \verb|jacobi|:带权的雅各比迭代,其中参数$\omega=\frac{2}{3}$
    \item \verb|Vcycle|与\verb|FMG|:两种迭代算法
    \item \verb|accuracy|:计算当前解的相对精度
    \item \verb|error_norm|:计算当前解的误差范数
    \item \verb|residual_norm|:计算当前解的残差范数
    \item \verb|solve|:制定$v_1$与$v_2$并进行求解
\end{itemize}

\subsection{参数与变量说明}
\subsubsection{bound\_conditon}
实现了三种边界条件:Dirichlet, Neumann,mixed。

其中,纯Neumann边界条件由于没有唯一解,计算时给定了区域中心点的具体点值使解唯一。

mixed条件在一维时为左端点满足Dirichlet条件,右端点满足Neumann边界条件,二维时为上下边界满足Dirichlet条件,左右边界满足Neumann边界条件

\subsubsection{stopping\_criteria}
实现了两种条件:max\_iteration,rela\_accuracy.

前者为限制最大迭代次数为st\_parm,后者为迭代至相对误差小于st\_parm。

但为了避免死循环和减少不必要的迭代,当迭代时误差的收敛速度小于1.01时,也会停止迭代。

\subsection{实现细节说明}
\subsubsection{restriction算子}
一维状态下的injection和full\_weighting和教材上的写法一致,二维状态下的injection是一维的简单推广。

特别说明以下二维的full\_weighting算子的实现方式:

\begin{itemize}
    \item 对于边界上的点,$I^{2h}_{i,j}=I^{h}_{i*2,j*2}$
    \item 对于中间区域的点,,$I^{2h}_{i,j}=\frac{I^{h}_{i*2,j*2}*4+I^{h}_{i*2,j*2-1}+I^{h}_{i*2,j*2+1}+I^{h}_{i*2-1,j*2}+I^{h}_{i*2+1,j*2}}{8}$
\end{itemize}
\subsubsection{interpolation算子}
一维状态下的linear和教材上的写法一致,quadratic即使用附近的三个粗网格上的点值插值得到二次函数计算细格点的值。

二维状态下的linear实现方式为:
\begin{itemize}
    \item 对于粗网格上的已有的点,$I^h_{i*2,j*2}=I^{2h}_{i,j}$
    \item 对于粗网格线上的点,$I^h_{i*2,j*2+1}=\frac{I^{2h}_{i,j}+I^{2h}_{i,j+1}}{2}$或者$I^h_{i*2+1,j*2}=\frac{I^{2h}_{i,j}+I^{2h}_{i+1,j}}{2}$
    \item 剩下的点,由于相邻的四个点已经可以计算,可以定义为$I^h_{i,j}=\frac{I^h_{i,j+1}+I^h_{i,j-1}+I^h_{i-1,j}+I^h_{i+1,j}}{4}$
\end{itemize}

二维状态下的quadratic实现为一维的简单推广,即先将x轴方向加细,再将y轴方向加细。虽然看起来有点水,但实现得到的效果显著(可参考后续数值测试部分)。

\section{程序测试结果}
我选择的测试用的一维函数为:
\begin{itemize}
    \item $f_1(x)=e^{sin(x)}-1$
    \item $f_2(x)=sin(x)$
    \item $f_3(x)=e^(x^2)$
\end{itemize}
二维函数为:
\begin{itemize}
    \item $f_1(x,y)=e^{x+sin(y)}$
    \item $f_2(x,y)=sin(3x+3y)$
    \item $f_3(x,y)=e^(x^3+y^3)$
\end{itemize}

data文件夹下是测试的具体结果:
\begin{itemize}
    \item \verb|cycle_iteration_convergence_rate_of_funX|:是一维函数$f_X$在所有参数组合下的每次迭代结果的误差范数、误差收敛速度、残差范数、误差收敛速度的结果。
    \item \verb|cycle_iteration_convergence_rate_of_fun2dX|:是二维函数$f_X$的对应结果
    \item \verb|four_grid_convergence_rate_of_funX|:是一维函数$f_X$在不同参数组合下,网格加细时的误差与残差结果和收敛速度
    \item \verb|four_grid_convergence_rate_of_fun2dX|:是二维函数$f_X$的对应结果
    \item \verb|cpu_time_of_mutigrid_and_LU_of_fun2dX|:是二维函数$f_X$在不同参数组合下,网格加细时的多重网格法和LU分解法求解的时间和时间的比值
\end{itemize}
\subsection{一维情形}
\subsubsection{{V-Cycle迭代的几个代表例子}}
\verb|cycle_iteration_convergence_rate_of_funX|文件中有完整的数据,此处挑选几个展示:
\begin{table}[H]
    \centering
    \caption{f1 , N=32  Dirichlet full\_weighting linear V\_cycle}
    \begin{tabular}{|c|l|l|l|l|}
    \hline
    Iteration & Error       & Ratio     & Residual    & Ratio     \\ \hline
    1 & 2.223e-02 & 0.000 & 9.132e-01 & 0.000\\ \hline 
    2 & 8.470e-04 & 26.250 & 2.589e-02 & 35.274\\ \hline 
    3 & 6.192e-05 & 13.679 & 7.783e-04 & 33.261\\ \hline 
    4 & 3.291e-05 & 1.882 & 2.561e-05 & 30.389\\ \hline 
    5 & 3.183e-05 & 1.034 & 8.781e-07 & 29.167\\ \hline 
    6 & 3.179e-05 & 1.001 & 3.084e-08 & 28.476\\ \hline 
    7 & 3.179e-05 & 1.000 & 1.093e-09 & 28.222\\ \hline 
    8 & 3.179e-05 & 1.000 & 3.892e-11 & 28.072\\ \hline 
    9 & 3.179e-05 & 1.000 & 1.408e-12 & 27.639\\ \hline 
    10 & 3.179e-05 & 1.000 & 1.697e-13 & 8.299\\ \hline 
    11 & 3.179e-05 & 1.000 & 1.381e-13 & 1.228\\ \hline 
    12 & 3.179e-05 & 1.000 & 1.306e-13 & 1.058\\ \hline 
    13 & 3.179e-05 & 1.000 & 1.113e-13 & 1.173\\ \hline 
    14 & 3.179e-05 & 1.000 & 8.585e-14 & 1.297\\ \hline 
    15 & 3.179e-05 & 1.000 & 1.280e-13 & 0.671\\ \hline 
    16 & 3.179e-05 & 1.000 & 1.460e-13 & 0.876\\ \hline 
    17 & 3.179e-05 & 1.000 & 1.118e-13 & 1.306\\ \hline 
    18 & 3.179e-05 & 1.000 & 1.266e-13 & 0.883\\ \hline 
    19 & 3.179e-05 & 1.000 & 1.193e-13 & 1.061\\ \hline 
    20 & 3.179e-05 & 1.000 & 1.276e-13 & 0.935\\ \hline     \end{tabular}
    \end{table}

    \begin{table}[H]
        \centering
        \caption{f2 , N=65536  Neumann full\_weighting quadratic V\_cycle}
        \begin{tabular}{|c|l|l|l|l|}
        \hline
        Iteration & Error       & Ratio     & Residual    & Ratio     \\ \hline
 1 & 1.334e-01 & 0.000 & 1.639e+03 & 0.000\\ \hline 
 2 & 6.688e-02 & 1.995 & 4.906e+01 & 33.417\\ \hline 
 3 & 2.571e-02 & 2.601 & 1.582e+00 & 31.011\\ \hline 
 4 & 8.213e-03 & 3.131 & 1.088e-01 & 14.536\\ \hline 
 5 & 2.303e-03 & 3.567 & 2.306e-02 & 4.719\\ \hline 
 6 & 5.890e-04 & 3.909 & 6.430e-03 & 3.587\\ \hline 
 7 & 1.413e-04 & 4.167 & 1.621e-03 & 3.966\\ \hline 
 8 & 3.246e-05 & 4.354 & 3.845e-04 & 4.217\\ \hline 
 9 & 7.237e-06 & 4.485 & 8.722e-05 & 4.408\\ \hline 
 10 & 1.582e-06 & 4.575 & 1.916e-05 & 4.553\\ \hline 
 11 & 3.414e-07 & 4.633 & 4.150e-06 & 4.617\\ \hline 
 12 & 7.375e-08 & 4.629 & 9.996e-07 & 4.151\\ \hline 
 13 & 1.803e-08 & 4.089 & 4.192e-07 & 2.385\\ \hline 
 14 & 5.369e-09 & 3.359 & 3.276e-07 & 1.280\\ \hline 
 15 & 1.900e-09 & 2.825 & 3.135e-07 & 1.045\\ \hline 
 16 & 9.988e-10 & 1.903 & 3.148e-07 & 0.996\\ \hline 
 17 & 5.938e-10 & 1.682 & 3.148e-07 & 1.000\\ \hline 
 18 & 4.559e-10 & 1.303 & 3.153e-07 & 0.998\\ \hline 
 19 & 4.794e-10 & 0.951 & 3.149e-07 & 1.001\\ \hline 
 20 & 4.246e-10 & 1.129 & 3.133e-07 & 1.005\\ \hline       \end{tabular}
    \end{table}
    \begin{table}[H]
        \centering
        \caption{f3 , N=512  mixed injection quadratic V\_cycle}
        \begin{tabular}{|c|l|l|l|l|}
        \hline
        Iteration & Error       & Ratio     & Residual    & Ratio     \\ \hline
        1 & 7.156e-02 & 0.000 & 3.961e+01 & 0.000\\ \hline 
        2 & 2.100e-02 & 3.407 & 1.391e+00 & 28.480\\ \hline 
        3 & 4.045e-03 & 5.192 & 5.999e-02 & 23.185\\ \hline 
        4 & 9.288e-04 & 4.355 & 3.995e-03 & 15.017\\ \hline 
        5 & 2.013e-04 & 4.614 & 5.381e-04 & 7.424\\ \hline 
        6 & 4.203e-05 & 4.790 & 1.048e-04 & 5.134\\ \hline 
        7 & 6.622e-06 & 6.347 & 2.260e-05 & 4.638\\ \hline 
        8 & 1.218e-06 & 5.439 & 4.992e-06 & 4.526\\ \hline 
        9 & 2.955e-06 & 0.412 & 1.105e-06 & 4.519\\ \hline 
        10 & 3.340e-06 & 0.885 & 2.448e-07 & 4.512\\ \hline 
        11 & 3.425e-06 & 0.975 & 5.425e-08 & 4.513\\ \hline 
        12 & 3.444e-06 & 0.995 & 1.202e-08 & 4.513\\ \hline 
        13 & 3.449e-06 & 0.999 & 2.665e-09 & 4.509\\ \hline 
        14 & 3.449e-06 & 1.000 & 6.014e-10 & 4.432\\ \hline 
        15 & 3.450e-06 & 1.000 & 1.616e-10 & 3.723\\ \hline 
        16 & 3.450e-06 & 1.000 & 8.564e-11 & 1.887\\ \hline 
        17 & 3.450e-06 & 1.000 & 7.376e-11 & 1.161\\ \hline 
        18 & 3.450e-06 & 1.000 & 7.663e-11 & 0.962\\ \hline 
        19 & 3.450e-06 & 1.000 & 7.482e-11 & 1.024\\ \hline 
        20 & 3.450e-06 & 1.000 & 7.301e-11 & 1.025\\ \hline       \end{tabular}
    \end{table}

可以发现三种边值条件下,使用不同的限制算子和插值算子,残差和误差都能较快地收敛较高的较低的水平,算法实现基本正确。
\subsubsection{}
\end{document} 