\documentclass[12]{article}%12pt即为*四号字
\usepackage{ctex}%引入中文包
\usepackage{graphicx}%插入图片的包
\usepackage{geometry}%设置A4纸页边距的包
\usepackage{url}
\usepackage{stfloats}
\usepackage{float}
\usepackage{amssymb}
\geometry{left=3.18cm,right=3.18cm,top=2.54cm,bottom=2.54cm}%设置页边距
\linespread{1}%设置行间距



\begin{document}
\begin{center}
    \LARGE\songti\textbf{微分方程数值解Project1作业报告} \\%标题
    \large\kaishu\textbf{褚朱钇恒\qquad 3200104144}%一般是我的姓名
\end{center}
\section{运行说明}
    本项目需要调用\verb|jsoncpp|与\verb|eigen3|库,故请在运行此项目前安装好这两个包。

    在\verb|project|目录下使用\verb|make|命令即可编译整个项目并得到实验报告。

\section{程序设计思路}
所有实现样条计算的相关代码都在头文件\verb|BVP.h|中,其中设计了一下几个类:
\subsection{Class Function}
其定义了\verb|()|,\verb|diff|,\verb|diff2|三个虚函数,分别用于函数求值,求导和求二阶导。

该基类有以下两个衍生类:
\begin{itemize}
    \item \verb|Class Polynomial| 用于存储一个多项式函数,形如$\Sigma Co_i(x-x_0)^i$
    \item \verb|Class B_spline_base| 用于存储一个$k$阶的B样条基函数,其控制点为$t$
\end{itemize}
\subsection{Class Interpolation}
定义了\verb|solve|,\verb|()|两个虚函数,分别用于插值系数计算和插值求值。
该基类有以下两个衍生类:
\begin{itemize}
    \item \verb|Class ppForm_interpolation| 用于使用多项式进行样条插值
    \item \verb|Class B_spline_base| 用于使用B样条基函数进行样条插值
\end{itemize}
以上两个衍生类可以在初始化时制定样条的阶数$order=1 or 3$,1对应线性样条,3对应三次样条($\mathbb{S}_3^2$);以及边界条件$condition=1 or 2 or 3$,m,1对应complete cubic spline,2对应cubic spline with specified second derivatives at its
end points,3对应natural cubic spline.

\section{作业题运行结果}
\end{document} 